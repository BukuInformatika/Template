
%% optional, but if you want to list author:

\introauthor{Rolly Maulana Awangga, S.T., M.T.}
{Informatics Research Center\\
Bandung, Jawa Barat, Indonesia}

Pada era disruptif  \index{disruptif}\index{disruptif!modern}
saat ini. git merupakan sebuah kebutuhan dalam sebuah organisasi pengembangan perangkat lunak.
Buku ini diharapkan bisa menjadi penghantar para programmer, analis, IT Operation dan Project Manajer.
Dalam melakukan implementasi git pada diri dan organisasinya.

Git ini sebenernya memudahkan programmer untuk mengetahui perubahan source codenya daripada harus membuat file baru seperti Program.java, ProgramRevisi.java,  ProgramRevisi2.java, ProgramFix.java. Selain itu, dengan git kita tak perlu khawatir code yang kita kerjakan bentrok, karena setiap developer bias membuat branch sebagai workspacenya.

Rumusnya cuman sebagai contoh aja biar keren\cite{awangga2018sampeu}.


\begin{equation}
ABC {\cal DEF} \alpha\beta\Gamma\Delta\sum^{abc}_{def}
\end{equation}
